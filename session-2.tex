%! Author = verwoerd
%! Date = 9-8-2023

% Preamble
\documentclass[11pt,pdf, aspectratio=169]{beamer}
\usetheme{metropolis}
\title{DAPC 2023 Training Sessions\\Session 2}
\author{Verwoerd}

% Packages
\usepackage{amsmath}
\usepackage[utf8]{inputenc}
\usepackage[T1]{fontenc}
\usepackage{graphicx}
\usepackage{tikz}
\usepackage{minted}
\usepackage[
  type={CC},
  modifier={by-sa},
  version={4.0},
]{doclicense}
\setsansfont{Fira Sans}
\usemintedstyle{manni}
\setminted{
  fontsize=\footnotesize,linenos,frame=lines, framesep=2mm
}
\usetikzlibrary{angles,quotes}

% Document
\begin{document}
  \maketitle
  \begin{frame}{Session 2}
    \begin{itemize}
      \item Team Tactics
      \item Utilizing the Test Session
      \item How to select problems
      \item Dealing with wrong submissions
      \item Solutions to the Ad-hoc and Math Problems
      \item Solving Sorting and Search Problems
    \end{itemize}
    \doclicenseThis
  \end{frame}


  \section{Team Tactics}
  \begin{frame}{General Tactics}
    \begin{itemize}
      \item Know each-others strength and weaknesses like:
      \begin{itemize}
        \item type of problems (math, geometry, search, strings, graphs etc.)
        \item debugging skills
        \item coding speed and accuracy
      \end{itemize}
      \item parallelize
      \item work on paper (e.g. pseudocode of flow diagram)
      \item debug on paper
    \end{itemize}
  \end{frame}
  \begin{frame}{Team Tactics}
    \begin{itemize}
      \item<1-> Plot of the contest: 3 contestants, 1 computer
      \item<2-> Several tactics how to divide the computer efficiently
      \item<3-> Shuffle Tactic
      \item<3-> Designated Tactic
      \item<3-> Pick and mix what works best for your team
    \end{itemize}
  \end{frame}
  \begin{frame}{Shuffle Tactic}
    \begin{itemize}
      \item Rotate around who sits behind the pc
      \item After submitting a problem switch around if someone has a solution
      \item Use when programming in different languages
    \end{itemize}
  \end{frame}
  \begin{frame}{Designated Tactic}
    \begin{itemize}
      \item Designated team member is in charge of the computer
      \item Prepare templates while other members read the problems
      \item 
    \end{itemize}
  \end{frame}


  \section{Solutions to the ad-hoc and math problems}
  \begin{frame}{Jabbing Jets}
    \begin{itemize}
      \item Source BAPC Preliminaries 2022
      \item Time limit: 1s
      \item Given $n$ concentric circles, find the maximal number of points on these circles such that the distance between any two points is at least $e$.
    \end{itemize}
    Original problem written by the BAPC 2022 jury and licensed under \doclicenseLongNameRef.

    \doclicenseImage

  \end{frame}
  \begin{frame}{Jabbing Jets}
    \begin{columns}
      \column{0.75\textwidth}
      \begin{itemize}
        \item Observation 1: time limit of 1s so looking for $\mathcal{O}(n)$ solution
        \item Observation 2: Because $r_{i+1}-r_i \geq e$ every circle can be considered separately.
        \item 2 points are divided by angle $\alpha$
        \item you can calculate $\alpha = 2 \arcsin{}(\frac{e}{2r})$
        \item The radius of a circle is $2\pi$ so the max number of points with distance $e$ can be calculated by\\
        $\left\lfloor\frac{2\pi}{\alpha}\right\rfloor \equiv \left\lfloor \frac{2\pi}{2 \arcsin{}(\frac{e}{2r})}\right\rfloor$
        \item Do this for every circle and sum the numbers
        \item Pitfall 1: if $2r < e$ then there can only be one point
        \item Pitfall 2: rounding can cause issues, add $0.5\cdot 10^{-6}$
      \end{itemize}
      \column{.25\textwidth}
      \newcommand{\myrad}{2cm}
\newcommand{\myang}{25}
\begin{tikzpicture}

  \coordinate (O) at (0,0);
  \coordinate (M) at (1.812, 0 );
  \coordinate (U) at (\myang:\myrad);
  \coordinate (L) at (-\myang:\myrad);
  \draw (O) node[circle, inner sep=1.5pt, fill] {} circle [ radius = \myrad];
  \draw (U) node[circle, inner sep=1.5pt, fill] {};
  \draw (L) node[circle, inner sep=1.5pt, fill] {};
  \draw (U)  --  node[midway, above] {$r$} (O) -- (L);
  \draw (U) -- node[midway, left] {$e$} (L);
  \draw pic [draw, -, angle radius=.75cm,"$\alpha$"] {angle = L--O--U};
%  \draw (O) -- (M);

\end{tikzpicture}

    \end{columns}
  \end{frame}
  \begin{frame}[containsverbatim]{Solution for Jabbing Jets}
    \inputminted{python}{code/session-1/python/dapc-j.py}
  \end{frame}
  \begin{frame}{Lots of Liquid}
    \begin{itemize}
      \item Source BAPC Preliminaries 2022
      \item Time limit: 1s
      \item Find the length of the side of a cube that contains all liquid.
    \end{itemize}
      Original problem written by the BAPC 2022 jury and licensed under \doclicenseLongNameRef.
      \doclicenseImage
  \end{frame}
  \begin{frame}{Lots of Liquid}
    \begin{itemize}
      \item Observation: Number of inputs is $10^9$, so we are looking for a $\mathcal{O}(n)$ solution
      \item The volume of a cube is $c^3$
      \item The length of cube with volume v  is $\sqrt[3]{v}$
      \item Sum all volumes of the cubes and calculate the length of the cube
      \item Print out \[\sqrt[3]{\sum_c^{i=0}c_i}\]
    \end{itemize}
  \end{frame}
  \begin{frame}[containsverbatim]{Lots of Liquid}
    \inputminted{python}{code/session-1/python/dapc-l.py}
    \inputminted{kotlin}{code/session-1/kotlin/dapc-l.kt}
  \end{frame}
  \begin{frame}{Bellevue}
    \begin{itemize}
      \item Source BAPC 2022
      \item Time limit: 1s
      \item  Given the profile of an island, find the point with the largest viewing angle of the sea.
    \end{itemize}
    Original problem written by the BAPC 2022 jury and licensed under \doclicenseLongNameRef.
    \doclicenseImage
  \end{frame}
  \begin{frame}{Bellevue}
    \begin{itemize}
      \item The answer is always an angle from the start or the end of the island to another point
      \item Calculate the angle from start and end to any other point
      \item print out the solution.\\
      \item alternatively calculate the convex hull
      \item the best angle is from either the first and second point in the hull or the last and previous.
    \end{itemize}
  \end{frame}
  \begin{frame}[containsverbatim]{Bellevue}
    \inputminted{python}{code/session-1/python/bapc-b.py}
  \end{frame}
  \begin{frame}{Equalising Audio}
    \begin{itemize}
      \item Source BAPC 2022
      \item Time limit: 4s
      \item  Given a list of frequencies, normalize it so the perceived loudness is \[\frac{1}{n} \sum^n_{i=1}a_i^2=x\]
    \end{itemize}
    Original problem written by the BAPC 2022 jury and licensed under \doclicenseLongNameRef.
    \doclicenseImage
  \end{frame}
  \begin{frame}{Equalising Audio}
    \begin{itemize}
      \item Observation time limit is high for I/O operations
      \item First calculate the current perceived loudness \[x_{cur} = \frac{1}{n} \sum^n_{i=1}a_i^2\]
      \item Then print out the frequency reduced by $\sqrt{\frac{x}{x_{cur}}}$, since
      \[\frac{1}{n} \sum^n_{i=1}\left(\sqrt{\frac{x}{x_{cur}}}a_i\right)^2 = \frac{1}{n} \sum^n_{i=1}\left(\sqrt{\frac{x}{x_{cur}}}\right)^2a_i^2 = \frac{x}{x_{cur}}\cdot \frac{1}{n}\sum^n_{i=1}a_i^2 = \frac{x}{x_{cur}}\cdot x_{cur} = x\]
      \item Pitfall: if the current perceived loudness ($x_{cur}$) is 0, then the result is \texttt{0 \ldots 0}
    \end{itemize}
  \end{frame}
  \begin{frame}[containsverbatim]{ Equalising Audio}
    \inputminted{python}{code/session-1/python/bapc-e.py}
    \inputminted{kotlin}{code/session-1/kotlin/bapc-e.kt}
  \end{frame}
  \begin{frame}{Failing Flagship}
    \begin{itemize}
      \item Source BAPC 2022
      \item Time limit: 1s
      \item  Compute the minimum angle in degrees between two wind directions.
    \end{itemize}
    Original problem written by the BAPC 2022 jury and licensed under \doclicenseLongNameRef.
    \doclicenseImage
  \end{frame}
  \begin{frame}{Failing Flagship}
    \begin{itemize}
      \item Convert the letters to degrees $d_1$ and $d_2$
      \item Execute the algorithm as described in the problem by adding and subtracting a delta, starting at $45\deg$
      \item Ever next letter represents a half of previous delta and either add or subtract it
      \item Be careful of overflow with NW
      \item print out the min($d_2-d_1, 360+d_1-d_2$)
    \end{itemize}
  \end{frame}
  \begin{frame}[containsverbatim]{Failing Flagship}
    \inputminted{python}{code/session-1/python/bapc-f.py}
  \end{frame}
\end{document}
